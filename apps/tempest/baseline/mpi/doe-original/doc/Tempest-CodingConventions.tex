\documentclass{article}

\usepackage{amsmath}
\usepackage{graphicx}
\usepackage{multicol}

\oddsidemargin 0cm
\evensidemargin 0cm

\textwidth 16.5cm
\topmargin -2.0cm
\parindent 0cm
\textheight 24cm
\parskip 0.5cm

\usepackage{fancyhdr}
\pagestyle{fancy}
\fancyhf{}
%\fancyhead[L]{AOSS Reference Sheet}
%\fancyhead[CH]{test}
\fancyfoot[C]{Page \thepage}

\newcommand{\vb}{\mathbf}
\newcommand{\vg}{\boldsymbol}
\newcommand{\mat}{\mathsf}
\newcommand{\diff}[2]{\frac{d #1}{d #2}}
\newcommand{\diffsq}[2]{\frac{d^2 #1}{{d #2}^2}}
\newcommand{\pdiff}[2]{\frac{\partial #1}{\partial #2}}
\newcommand{\pdiffsq}[2]{\frac{\partial^2 #1}{{\partial #2}^2}}
\newcommand{\topic}{\textbf}
\newcommand{\arccot}{\mathrm{arccot}}
\newcommand{\arcsinh}{\mathrm{arcsinh}}
\newcommand{\arccosh}{\mathrm{arccosh}}
\newcommand{\arctanh}{\mathrm{arctanh}}

\title{\Huge \textbf{Tempest Framework: Coding Conventions}}
\author{\Large Paul Ullrich}
\date{November 22nd, 2013}

\begin{document}

\section{Tempest Framework: Coding Conventions}

\begin{itemize}
\item Variable names should following Hungarian notation conventions.

\item Indentation shall be strictly performed with tabs.

\item All \texttt{if} blocks must be accompanied by braces.

\item One command per line except in rare circumstances (such as an assignment of a pair of variables).  The author should be careful when including more than one command per line to ensure that readability of the code is not lost.

\item Pointers and references shall be identified with whitespace as follows:
\begin{verbatim}
double * pTest = &test;
double & dTest = test;
\end{verbatim}
\end{itemize}

\end{document}