\documentclass{article}
\usepackage{algorithm}
\usepackage{algorithmicx}
\usepackage{algpseudocode}
\usepackage{graphicx}
\usepackage{amsmath}

\interfootnotelinepenalty=10000
\begin{document}

\title{The XSBench Mini-App - A Discussion of Theory}

\author{John Tramm\\jtramm@mcs.anl.gov}

\maketitle

\begin{abstract}

A kernel has been developed that mimics the most computationally expensive steps of a robust nuclear reactor core Monte Carlo particle transport algorithm -- the calculation of macroscopic cross sections. This document explains the theoretical basis of the kernel and outlines the current topics of research the kernel facilitates.

\end{abstract}

\section{Introduction}

\subsection{The Reactor Simulation Problem}

The computerized simulation of nuclear reactors is a well established field, with origins dating back to the early years of digital computing. Traditional reactor simulation techniques aim to solve the diffusion equation for a given material geometry and starting (source term) neutron distribution within the reactor. This is done in a deterministic fashion using well developed numerical methods. However, deterministic methods suffer from a number of problems; namely, a significant number of approximations and simplifications must be made which results in non-trivial margins of error when compared to real-world results.

An alternative method, Monte Carlo (MC) simulation, simulates the path of a particle (neutron) as it travels through the reactor core. As many particle histories are simulated, a picture of the full distribution of neutrons within the reactor core is developed. Furthermore, the methodologies utilized by MC simulation require very few assumptions, resulting in incredibly accurate results. The downside to this method, however, is that a huge number of neutron histories must be run in order to achieve an acceptably low variance in the results. Specifically, the correlation between runtime and variance is governed by the central limit theorem, which states that the variance of the sample mean is inversely proportional to the number of particle histories realized, i.e., :

\begin{align*}
\sigma^2\propto\frac{1}{N}
\end{align*}

\subsection{OpenMC}

OpenMC is a Monte Carlo particle transport simulation code focused on neutron criticality calculations. It is capable of simulating 3D models based on constructive solid geometry with second-order surfaces. The particle interaction data is based on ACE format cross sections, also used in the MCNP and Serpent Monte Carlo codes.

OpenMC was originally developed by members of the Computational Reactor Physics Group at the Massachusetts Institute of Technology starting in 2011. Various universities, laboratories, and other organizations (including CESAR) now contribute to the development of OpenMC.

\subsection{XSBench}

The XSBench proxy app models the most computationally intensive part of a typical MC transport algorithm - the calculation of macroscopic neutron cross sections - a kernel which accounts for around 85\% of the total runtime of OpenMC. The essential computational conditions and tasks of fully featured MC neutron transport codes are retained in the mini-app, without the additional complexity of the full application. This provides a much simpler and more transparent platform for determining performance benefits resulting from a given hardware feature or software optimization.

XSBench is under ongoing development by members of the Center for the Exascale Simulation of Advanced Reactors (CESAR) group at Argonne National Laboratory. Development began in 2012.

\section{Algorithm}

\subsection{Reactor Model}

When carrying out reactor core analysis, the geometry and material properties of a postulated nuclear reactor must be specified in order to define the variables and scope of the simulation model. For the purposes of XSBench, we use a well known community reactor benchmark known as the Hoogenboom-Martin model \cite{Hoogenboom}. This model is a simplified analog to a more complete, ``real-world'' reactor problem, and provides a standardized basis for discussions on performance within the reactor simulation community. XSBench recreates the computational conditions present when fully featured MC neutron transport codes (such as OpenMC) simulate the Hoogenboom-Martin reactor model, preserving a similar data structure, a similar level of randomness of access, and a similar distribution of flops and memory loads.

\subsection{Neutron Cross Sections}

The purpose of an MC particle transport reactor simulation is to gain useful data about the distribution and generation rates of neutrons within a nuclear reactor. In order to achieve this goal, a large number of neutron  lifetimes are simulated by tracking the path and interactions of a neutron through the reactor from its birth in a fission event to its escape or absorption, the latter possibly resulting in another fission event.

Each neutron in the simulation is described by three primary factors: its spatial location within a reactor's geometry, its speed, and its direction. At each stage of the transport calculation, a determination must be made as to what the particle will do next. For example, some possible outcomes include uninterrupted continuation of free flight, collision with another atom, or fission with a fissile material. The determination of which event occurs is based on a random sampling of a statistical distribution that is determined by empirical material data stored in main memory. This data, called {\em neutron cross section data}, represents the probability that a neutron of a particular speed (energy) will undergo some particular interaction when it is inside a given type of material. To account for neutrons across a wide energy spectrum and materials of many different types, the structure that holds this cross section data must be very large. In the case of the simplified Hoogenboom-Martin benchmark roughly 5.6 GB\footnote{We estimate that for a robust depletion calculation in excess of 100 GB of cross section data would be required.} of data is required.

\subsection{Data Structure}

A nuclide is a type of atom described by a specific number of neutrons and protons. A given element can have many different nuclides, such as Uranium-235 or Uranium-238. A material in the reactor model is composed of a mixture of nuclides. For instance, the ``reactor fuel'' material might consist of several hundred different nuclides, while the ``pressure vessel side wall'' material might only contain a dozen or so. In total, there are 12 different materials and 355 different nuclides present in the modeled reactor.

For each nuclide, an array of nuclide grid points are stored as data in main memory. Each nuclide grid point has an energy level, as well as five various cross sections (corresponding to five different particle interaction types) for that energy level. The arrays are ordered from lowest to highest energy levels. Each nuclide has a different ``grid spacing,'' i.e., the number, distribution, and granularity of energy levels varies between nuclides. One nuclide may have hundreds of thousands of grid points clustered around lower energy levels, while another nuclide may only have a few hundred grid points distributed across the full energy spectrum. This obviates straightforward approaches to uniformly organizing and accessing the data.

In order to increase efficiency of access, the algorithm utilizes another data structure, called the \emph{unionized energy grid}, as described by Lepp\"{a}nen \cite{Leppanen} and Romano \cite{Romano:OpenMC}. The unionized grid facilitates fast lookups of cross section data from the nuclide grids. This structure is an array of grid points, consisting of an energy level and a set of pointers to the closest corresponding energy level on each of the different nuclide grids.

\begin{table}[h]
\caption{XSBench Data Structure Summary}
\label{tab:XS}
\begin{center}
\begin{tabular}{ | c | c | }
    \hline
    Nuclides Tracked & 355\\ \hline
    Total \# of Energy Gridpoints &  4,012,565\\ \hline
    Cross Section Interaction Types & 5 \\ \hline
    Total Size of Cross Section Data Structures & 5.6 GB \\ \hline
\end{tabular}
\end{center}
\end{table}

\subsection{Access Patterns}

In a full MC neutron transport application, the data structure is accessed each time a macroscopic cross section needs to be calculated. This happens anytime a particle changes energy (via a collision) or crosses a material boundary within the reactor. These macroscopic cross section calculations are extremely common in the MC transport algorithm, and the inputs to them are effectively random. For the sake of simplicity, XSBench was written ignoring the particle tracking aspect of the MC neutron transport algorithm and instead isolates the macroscopic cross section lookup kernel. This provides a large reduction in program complexity while retaining similarly random input conditions for the macroscopic cross section lookups via the use of a random number generator.

In XSBench, each macroscopic cross section lookup consists of two randomly sampled inputs: the neutron energy and the material. Given these two inputs, a binary (log n) search is done on the unionized energy grid for the given energy. Once the correct entry is found on the unionized energy grid, the material input is used to perform lookups from the nuclide grids present in the material. Use of the unionized energy grid means that binary searches do not need to be performed on each individual nuclide grid. For each nuclide present in the material, the two bounding nuclide grid points are found using the pointers from the unionized energy grid and interpolated to give the exact microscopic cross section at that point.

All calculated microscopic cross sections are then accumulated (weighted by their atomic density in the given material), which results in the macroscopic cross section for the material. The calculation of macroscopic cross sections is depicted in Algorithm~\ref{alg:code}.

\begin{algorithm}
\caption{Macroscopic Cross Section Lookup}
\label{alg:code}
\begin{algorithmic}[1]

\State{$R(m_p,E_p)$}
\Comment{randomly sample inputs}
\State{\em{Locate $E_p$ on Unionized Grid}}
\Comment{binary search}
\For{$n \in m_p$}
	\Comment{for each nuclide in input material}
	\State{$\sigma_a \leftarrow n,E_p$}
	\Comment{lookup bounding micro xs's}
	\State{$\sigma_b \leftarrow n,E_p+1$}
	\State{$\sigma \leftarrow \sigma_a,\sigma_b$}
	\Comment{interpolate}
	\State{$\Sigma \leftarrow \Sigma + \rho_n\cdot\sigma$}
	\Comment{accumulate macro xs}
\EndFor

\end{algorithmic}
\end{algorithm}

In theory, one could ``pre-compute'' all macroscopic cross sections on the unionized energy grid for each material. This would allow the algorithm to run much faster, requiring far fewer memory loads and far fewer floating point operations per macroscopic cross section lookup. However, this would assume a static distribution of nuclides within a material. In practice, MC transport algorithms will need to track the burn-up of fuels, as well as heterogeneous temperature distributions within the reactor itself. This means that concentrations are dynamic, rather than static, therefore necessitating the use of the more versatile data model deployed in XSBench.

We have verified that XSBench faithfully mimics the data access patterns of the full MC application under a broad range of conditions. The runtime of full-scale MC transport applications, such as OpenMC, is 85\% composed of macroscopic cross section look ups. Within this process, XSBench is virtually indistinguishable, as the same type and size of data structure is used, with a similarly random access pattern and a similar number of floating point operations occurring between memory loads. Thus, performance analyses done with XSBench will provide results applicable to the full MC neutron transport algorithm, while being easier to interpret.

\section{Early Results}

We have successfully run XSBench on a wide variety of shared-memory architectures. Below is a list of systems XSBench has been deployed on:

\begin{enumerate}
\item IBM BlueGene/Q \emph{Mira} - 16 cores
\item Intel Xeon E5620 - 8 cores
\item Intel i7-3615QM - 4 cores
\item Intel i7-2600k - 4 cores
\end{enumerate}

Strong scaling results vary significantly between systems. However, generally speaking, we have seen approximately 76\% scaling on the Intel systems, and approximately 95\% scaling on BlueGene/Q, where scaling is defined as the ratio of realized performance using all cores on the node to the performance of a single core multiplied by the number of cores on the node, i.e.:

\begin{align*}
Scaling = \frac{P_n}{P_1 \times n}
\end{align*}

Where $P_n$ is the performance on all $n$ cores, and $P_1$ is the performance on a single core.

\section{Topics of Investigation}

\subsection{Problem Statement}

We are looking to shed light on why, precisely, we are losing performance as cores are added to a node. We know that we are not memory bandwidth constricted, however, we suspect that we are generally limited by memory latency/contention issues. We want to know which exact subsystems are responsible for this. For instance, what is it about the architectures of BG/Q vs intel Sandy Bridge that result in such drastic differences in scaling? Why is BG/Q at 96\% efficiency at 1 thread per core, whereas the intel chip is at 76\% efficiency at 1 thread per core? We really want to know exactly where these losses are coming from, so that we can make an intelligent prediction as to how this type of code will scale when there are hundreds of cores per node. Furthermore, we want to know if there are any changes we can make, either algorithmically or in terms of data structures, that could potentially get around the current hardware bottlenecks.

We would be perfectly happy with 76\% or 96\% scaling if that was as bad as things would ever get in the future. However, given the shape of the scaling curves (downwards), we want to be certain about what exactly is causing degradation in the first place, and whether or not these causes will result in continually diminishing returns as we add cores to a node in future systems.

\subsection{Avenues of Research}

We have experimented already in several areas. We have added dummy flops in between memory loads, and found that this tends to improve scaling. We think this effect can be explained by the extra flops masking the latency to main memory. This helps to isolate memory latency as a probably bottleneck.

We have also experimented with making each of the memory loads much larger, in terms of number of bytes per load. This did not have any significant effect on scaling. We believe that this demonstrates that the algorithm is not (currently) memory bandwidth dependent, because the extra bandwidth required to transmit the larger memory loads should immediately cause scaling degradation if there is not enough available bandwidth. Since there was no degradation, there must be extra bandwidth still available.

Additionally, we have experimented with hardware threads and CPU vector directives (such as intel SSE or IBM QPX) as possible avenues for improvement. We found that hardware threads are beneficial to the performance of the algorithm - even more so than most other types of HPC codes. We are close to getting results from the vector directives approach.

We have also been attempting to construct a compelling narrative from the data that explains what exact hardware subsystems currently bottleneck the code and result in the scaling degradation observed. We have used PAPI to generate performance counter data, but have so far been unable to coax the data into any coherent form capable of shedding light on our situation. I.e., raw counter data isn't of much use unless we have a way to make sense out of it.

\section{Future Work}

There are additional capabilities that do not yet exist in full-scale MC neutron transport algorithms, such as on-the-fly doppler broadening to account for the material temperature dependence of cross sections, that we plan on adding to XSBench for experimentation with various hardware architectures and features. Furthermore, we want to investigate cross section data compression techniques, specifically by reducing the data into resonance regions, in the hopes of quantifying the trade offs (in terms of decreased DRAM memory footprint/traffic vs. the increase in CPU flop requirements) of such compression techniques.

\section*{Acknowledgments}

This work was supported by the Office of Advanced Scientific Computing Research, Office of Science, U.S. Department of Energy, under Contract DE- AC02-06CH11357. The submitted manuscript has been created by the University of Chicago as Operator of Argonne National Laboratory (�Argonne�) under Contract DE-AC02-06CH11357 with the U.S. Department of Energy. The U.S. Government retains for itself, and others acting on its behalf, a paid-up, nonexclusive, irrevocable worldwide license in said article to reproduce, prepare derivative works, distribute copies to the public, and perform publicly and display publicly, by or on behalf of the Government.

\bibliographystyle{IEEEtran}
\begin{thebibliography}{1}

\bibitem{Hoogenboom}J.E. Hoogenboom, W.R. Martin, B. Petrovic, ``The Monte Carlo performance benchmark test � aims, specifications and first results,'' International Conference on Mathematics and Computational Methods Applied to Nuclear Science and Engineering, Rio de Janeiro, Brazil (2011).

\bibitem{Leppanen}Jaakko Lepp\"{a}nen, ``Two practical methods for unionized energy grid construction in continuous-energy Monte Carlo neutron transport calculation,'' Annals of Nuclear Energy, Volume 36, Issue 7, July 2009, Pages 878-885.

\bibitem{Romano:OpenMC}Paul K. Romano, Benoit Forget, ``The OpenMC Monte Carlo particle transport code,'' Annals of Nuclear Energy, Volume 51, January 2013, Pages 274-281.

\bibitem{Romano:Scaling}Paul K. Romano, Benoit Forget, Forrest Brown, ``Towards Scalable Parallelism in Monte Carlo Particle Transport Codes Using Remote Memory Access,'' Progress in Nuclear Science and Technology, Volume 2, October 2011, Pages 670-675.

\bibitem{Saini}S. Saini, H. Jin, R. Hood, D. Barker, P. Mehrotra, R. Biswas, The impact of hyper-threading on processor resource utilization in production applications, 18th International Conference on High Performance Computing (HiPC), Bangalore, India, Dec. 2011.

\bibitem{Siegel} Andrew Siegel, Kord Smith, Paul Romano, Ben Forget, Kyle Felker, ``Multi-core performance studies of a Monte Carlo neutron transport code,'' International Journal of High Performance Computing Applications. (Under Review)

\bibitem{Tramm} John Tramm, Andrew Siegel, ``Memory Bottlenecks and Memory Contention in Multi-Core Monte Carlo Transport Codes,'' Joint International Conference on Supercomputing in Nuclear Applications and Monte Carlo, 2013. (Submitted, Pending Acceptance \& Review. Not yet Published.)

\end{thebibliography}

\end{document}

%\subsection{Subsection Heading Here}
%Subsection text here.
%\subsubsection{Subsubsection Heading Here}
%Subsubsection text here.

